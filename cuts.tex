\begin{figure*}
\begin{verbatim}
{
  "name": "Peter van Hardenberg"
  "avatar": "hypermerge:/DaWBgZASxr6ksJHtsx5FC7Ug9hANRgmYxTNaHX9J3oG5?contentType=image",
  "devices": [
    "hypermerge:/VKwCjhq9LY3w3wr8tqF2JoHpqafs1zxLmMMCNM1aUP9?contentType=device",
    "hypermerge:/KQoZYf8c31dgpasC7iaXAvrZJtRzQL7Q1dMyAJRNAyQ?contentType=device"
  ]
}
\end{verbatim}
\caption{JSON document representing a PushPin user, containing their name, avatar image, and links to their devices.}
\label{fig:user-json}
\end{figure*}

In the early days of the web, browsers traversed static HTML. As time went on, web applications became increasingly powerful and a separation of responsibilities between server-side and client-side code emerged. Conceptually, a Single Page App (SPA) is a static program which runs in the client's browser and makes requests to server-side APIs for the data it requires to respond to user input. (Practically speaking it is common to blur these distinctions, such as by doing some server-side rendering of initial state.) Managing the state of the client-side application in an ad-hoc way can be difficult, managing the currently rendered state of the browser, overlapping web requests, user input, and so on. 


\subsection{Users}

Opaque cryptographic hashes are difficult to recognize and remember, subject to transcription errors, and don't communicate origin or authorship in the way that traditional web URLs can. Within PushPin, we don't have a centralized account system, but we did want to have a way for users to identify themselves to other participants and share content with those they've met before. We also felt it was important to include in-app document sharing that relied as little as possible on copying and pasting links outside the application.

First, in PushPin we model users as regular documents. Users have a name and a contact photo. This document also stores information about what documents users are sharing as an "outbox". An outbox is an inbox, in reverse. It is a list of documents you'd like other people to be able to access, and the data structure is simple. The "shares" field on a user document is a map from user profile document IDs to a list of URLs being shared. We chose this structure to allow users to post shares for people whose identities they have but who they have not yet connected with. Given Bob's contact information, Alice can post a share for him, then go offline. As long as that data is hosted by some other node in the graph (either a storage peer or a collaborator), when Bob requests Alice's profile document, he can download the relevant data.

Of course, the first and most obvious problem is that these shares should not be plain-text. We solve this problem by obfuscating shares inside a sealed-box encryption payload based on the public key shared by the recipient. This public key is important. If it were to be replaced by another user, that person would hijack access to all documents being sent to the original recipient.

This problem led us to our first experiments in implementing something like "merge permissions" in PushPin. Briefly, this special key is protected by a designation that causes all other authors' writes to the key to be ignored. This limited nod to the notion that users should make informed decisions about which changes they accept was forced by concern for the security of our beta testing group and does not yet reflect a fully considered design.

% On cloud peers

We feel this is an important distinction from the notion of "pinning" found in some other systems such as IPFS. In general, PushPin is a creative tool oriented towards storing data created locally by the user or by one of their collaborators. The default should be to preserve everything! In contrast, IPFS is oriented towards generic binary object storage, and so does not assume that interacting with an object would imply wanting to retain it indefinitely.

One of the goals for pushpin-peer was that the UI had to be self-hosted, which is to say that all the interactions with it should occur within PushPin and not in some other web administration tool. When pushpin-peer starts up, it prints out a URL that users can load from their PushPin. Users interact with their cloud peer from within PushPin by making edits to PushPin's adminstration document. The pushpin-peer daemon monitors that document for changes and takes actions based on what it finds. This approach was first pioneered in our work on pixelpusher, and has returned with greater sophistication here. Most importantly, pushpin-peer uses similar techniques to the user document sharing approach described above to allow clients to write private URLs to a document other peers may read without disclosing their contents.

% On access revocation

Unfortunately, the stability of these URLs is somewhat problematically strong. The URLs function as a sort of security capability, and once a client has seen the URL, there is no real mechanism in the current system to revoke or rotate it. Further, while we believe CRDTs are an exciting area to explore new collaboration models \cite{Pixelpusher} we have not pursued this effort in PushPin. We have often described our approach in PushPin as "radical mandatory collaboration". Once a document has been shared, a client will always accept all updates to it from all users with the URL anywhere forever. This is, to be clear, not a good design, but rather a deliberate absence of design. 

\subsection{Online/Offline State}

PushPin is designed to support not only collaboration between people, but also synchronisation across multiple devices belonging to the same user. A user document includes a list of devices they are associated with, and users are considered online if any of those devices are reachable. Availability of a user's own devices is also important. A user with no other online devices is making changes without a backup. If another of their devices is online, their work can be synchronised there and become more durable. It can be important to a user to understand their relative state to a variety of users. If PushPin has sent a fresh copy of a document to an offline user just a moment ago, that's meaningfully different than if the user does not appear to have used the application in weeks. 

% cut discussion of automerge performance


% this was section 6

\section{Lessons from Implementing PushPin}\label{sec:lessons}

Our goal in building PushPin was to explore whether it was possible to apply recent developments in CRDTs and peer-to-peer networking with modern web development technologies to produce an application with several properties.

Those properties are:
 - long-now reliability, by not relying on external network services
 - real-time collaboration throughout the software
 - consistent performance with low latency
 - a high degree of usability for a general audience 

Because of the first and second goals Peer-to-peer systems have a number of unique challenges versus centralised systems, including:

\begin{itemize}
	\item no single source of truth, no single authority

	Without a centralised server hosting an authoritative copy, every copy of the data becomes, in some sense, authoritative. This implies that we must have some kind of robust merging strategy at the core of our design.
	
	\item devices get out of sync

    Because we insist that each device is always able to make progress without communication with other nodes, it is vitally important to both do our best to synchronise our various devices, but also to communicate synchronisation state. What were the last changes received from our laptop? Did the changes we made locally get uploaded to a server before the network connection was lost?  

    \item connectivity is non-boolean

    With a centralised system a user is either offline or online. The central node routes all data. With a peer-to-peer system is is entirely possible that you might be connected with another user around the world but due to the complexities of computer networking, disconnected from a user in the same room. (Perhaps their wifi connection is offline.) 

    \item changes can come from anywhere at any time

    Without the centralised system approving or rejecting changes against a monotonically incrementing API version, it's possible that other clients are running older, newer, or just different versions of the program. How do we robustly manage change over time?
    
    \item different devices might be running different versions of the software, and they need to interoperate cleanly
\end{itemize}

It is a sad truth that peer-to-peer software has earned a rather poor reputation not just for reliability but also as a vector of illicit activity. Because of this, router vendors and network administrators both collude to restrict peer-to-peer communications and simply fail to support them.

what works, what's not working so well?

\begin{itemize}
    \item FRP render loop works great -- taking a functional transform of a document state to application view means never worrying about where the update came from or how to render it
	\item storing \& distributing append-only logs is very simple and robust
    \item peer-to-peer networking is highly problematic
    \begin{itemize}
	    \item webrtc (not very good, requires centralized assets)
		\item DHTs (reliability, privacy)
		\item centralized services (fragility)
		\item router configuration issues
	\end{itemize}
	\item much performance work has been done
	\begin{itemize}
	    \item architect to separate render \& computation of CRDT operations
	\end{itemize}
	\item conflict resolution surprisingly unproblematic
	\item merge \& collaboration UX is largely unexplored (could show a pixelpusher slide \& briefly explain problems)
	\item what kind of identity, privacy, sharing features are important / feasible here?
	\item html is a pretty rough application development platform
	\item no real mobile support
	\begin{itemize}
	    \item what's the cross-platform story?
	\end{itemize}
	\item URLs containing long alphanumeric strings (corresponding to cryptographic keys) are a poor user experience. We would like to minimise the degree to which users have to copy and paste them.
\end{itemize}

By creating documents out of trustworthy append-only logs with stable names, we enable links and stably-named collaborative documents. A variety of state-of-the-art networking strategies overcomes the limitations of commonly available NAT routers in the field. Unfortunately, these techniques are not reliable enough to eliminate all traffic proxy requirements, and so we employ optional, but helpful network-persistent peers which act as relay points and data caches to improve reliability. (Importantly, they are not a central resource but can be run by any motivated network participant.) Last, we discuss some of our early work in improving discovery strategies for scalability without sacrificing specificity and some of the problems with current distributed hash table implementations today.
